\documentclass[A4,12pt]{article}

\usepackage[italian]{babel}
\usepackage{amsmath,amsthm}
\usepackage{amsfonts}
\usepackage{amssymb}
\usepackage[italian]{babel}
\usepackage{fancyhdr}
\usepackage[top=3cm, bottom=4cm, left=2.5cm, right=2.5cm]{geometry}
\usepackage{xcolor}
\usepackage{enumitem}
\usepackage{scalerel}
\usepackage{multicol}
\usepackage[thinc]{esdiff}
\usepackage{mathrsfs}
\usepackage{mathtools}
\usepackage{dsfont}



\newcommand{\ti}[1]{{\sc \textcolor{teal}{\underline{#1}:}}}

\newcommand{\df}{\ti{Def}\ }
\newcommand{\os}{\ti{Oss}\ }
\newcommand{\pr}{\ti{Prop}\ }
\newcommand{\tr}{ {\large \ti{th}}\ }
\newcommand{\cor}{\ti{Cor}\ }
\newcommand{\ntz}{\ti{Not}\ }
\newcommand{\es}{\ti{Es}\ }
\newcommand{\lm}{\ti{Lemma}\ }
\newcommand{\R}{\mathbb{R}}
\newcommand{\Z}{\mathbb{Z}}
\newcommand{\C}{\mathbb{C}}
\newcommand{\Q}{\mathbb{Q}}
\newcommand{\N}{\mathbb{N}}

\newcommand{\Cdiff}{\C\textit{-diff}}
\newcommand{\Hc}{\mathscr{H}}
\newcommand{\va}[1]{\lvert #1 \rvert}
\newcommand{\nr}[1]{\lVert #1 \rVert}
\newcommand{\grgen}[1]{< \hspace{-1mm} #1 \hspace{-1mm} >}
\newcommand{\ZnZ}{\Z/n\Z}

\renewcommand{\subset}{\subseteq}
\newcommand{\defeq}{\vcentcolon=}
\newcommand{\eqdef}{=\vcentcolon}
\newcommand{\erre}{\mathbb{R}}
\newcommand{\errex}{\erre \cup \{\pm \infty\}}
\newcommand{\ci}{\mathbb{C}}
\newcommand{\erren}{\mathbb{R}^N}
\newcommand{\zetabb}{\mathbb{Z}}
\newcommand{\enne}{\mathbb{N}}
\newcommand{\spmis}{(\Omega,\mathcal{M})}
\newcommand{\spdimis}{(\Omega,\mathcal{M},\mu)}
\newcommand{\compl}[1]{\mathcal{C}_{#1}}
\newcommand{\sigal}{\mathcal{M}}
\newcommand{\parti}[1]{\mathcal{P}(#1)}
\newcommand{\Bigcup}{\bigcup\limits}
\newcommand{\Bigcap}{\bigcap\limits}
\newcommand{\Sup}{\sup\limits}
\newcommand{\Max}{\max\limits}
\newcommand{\Min}{\min\limits}
\newcommand{\Inf}{\inf\limits}
\newcommand{\Limsup}{\limsup\limits}
\newcommand{\Liminf}{\liminf\limits}
\newcommand{\Sum}{\sum\limits}
\newcommand{\Prod}{\prod\limits}
\newcommand{\Int}{\displaystyle\int}
\newcommand{\Lim}{\lim\limits}
\newcommand{\mustar}{\mu^\ast}
\newcommand{\indic}{\mathds{1}}
\newcommand{\norm}[1]{\lVert #1 \rVert}
\newcommand{\acts}{\curvearrowright}
\newcommand{\notimplies}{\;\not\!\!\!\implies}
\newcommand{\dpartial}[2]{\dfrac{\partial #1}{\partial #2}}
\newcommand{\then}{\Rightarrow}
\newcommand\restr[2]{#1_{|_{#2}}}

\newcommand{\cmp}[1]{(#1_1,...,#1_n)}
\newcommand{\B}{\mathcal{B}}
\usepackage{csquotes}
%for i = 1:100 println("                \\item $i→ ") end

\newcommand{\Ginsmth}{  G\!\mathbf{-insiemi}}
\newcommand{\Gins}{$ G $-ins.}
\renewcommand{\iff}{\Leftrightarrow}
\newcommand{\psy}{$ p $-sylow }
\begin{document}
	
	\begin{itemize}[noitemsep]
		\item 1→ Qual è la definizione di azione sinistra? → Dato un gruppo $ G $ e un insieme $ X $ definisco \textsc{azione di $X$ su $ G $ } un'applicazione $ A:G\times X\to X $ che soddisfa: \begin{enumerate}[noitemsep]
			\item $ A(e,x) = x\ \forall x \in X $
			\item $ A(g,A(h,x)) = A(gh,x)\ \forall g,h\in G,\ \forall x\in X$
		\end{enumerate} 
		Dico che $ G $ \textsc{agisce su} $ X $, $ G\acts X$
		\item 3→  Fornisci un esempio di azione su uno spazio vettoriale $V$ → È sufficiente prendere $A:G\times V\to V: (f,v)\mapsto f(v)$
		\item 3.01→  Illustra come $S_X$ agisce su $X$ → L'azione è $ g\cdot x := g(x) $ 
		\item 4→ Enuncia il lemma di caratterizzazione delle azioni di gruppo come morfismi nel gruppo simmetrico → Data un'azione $ A:G\times X \to X:(g,x)\mapsto g\cdot x $, posso definire un morfismo $ \alpha: G\to S_X $ ponendo $ \alpha(g) $ la funzione $X\to X: x\mapsto A(g,x) $. (Cioè quindi $ \alpha(g)(x):=A(g,x) $) (da dimostrare che $ \alpha $è ben definita).\\ Viceversa, dato $ \alpha:G\to S_X $ morfismo, posso definire un'azione $ A $ come segue: $ A(g,x) := \alpha(g)(x) $ (da dimostrare che è un'azione)
		\item 5→  Qual è la definizione di insieme $G$-invariante? →  Se $G\acts X$, un sottoinsieme $Y\subseteq X$ è \textsc{invariante} quando $g\cdot y = y\ \forall y \in Y $ 
		\item 6→ Quali invarianti puoi trovare per $ \operatorname{SO}(3)\acts \R^3 $? → Le sfere sono $ G $-invarianti 
		\item 7→ Come si comportano i sottogruppi di un gruppo che agisce su un insieme? → Agiscono anche loro sullo stesso inseme con la restrizione dell'azione
		\item 8→ Qual è la definizione di $ G $\textsc{-orbita} di $ x\in X $ ? → È $ Gx = \{g\cdot x \forall g \in G\} $
		\item 9→ La collezione delle orbite di un'azione come si rapporta all'insieme su cui agisce l'azione? → Data $ G\acts X $, le sue orbite formano una partizione di $ X $ 
		\item 10→ Essere in nella sessa orbita è {{c1:: una relazione d'equivalenza}} → clz 
		\item 11→ Qual è la definizione di $ X/G $ ? → $ X/G:= X/\sim $, con $ \sim $ relazione di equivalenza su $ X $ di appartenenza alla stessa orbita.
		\item 12→ Qual è la definizione di \textsc{stabilizzatore} di $ x\in X $? → È $ G_x :=\{g\in G\mid g\cdot x = x\}$ 
		\item 13→ Come sono in relazione gli stabilizzatori di due elementi sulla stessa orbita? → Se $ x$ e $ y $ sono sulla stessa orbita, allora $ G_x $ e $ G_y $ sono coniugati.
		\item 14→ Qual è la definizione di azione transitiva? → $ G\acts X $ è \textsc{transitiva} quando $ \forall x,y\in X\ \exists g\in G:\ g\cdot x = y$ (cioè ho un'unica orbita, $ Gx = X $)
		\item 15→ $ SO(3)\acts S^2 $ è {{c1::transitiva }} → clz
		\item 16→ Qual è la definizione di spazio omogeneo? → Data $ G\acts X $ azione, se è transitiva dico che $ X $ è uno \textsc{spazio omogeneo} per $ G $ 
		\item 17→ Esibisci un'azione di un gruppo sul quoziente per un sottogruppo → In generale: se $ H\leq G, X:= G/H $ ho che $ G\acts G/H:\ g\cdot aH:= gaH $
		\item 18→ Qual è la definizione di un $ G $-\textsc{insieme}? → è un insieme $ X $ su cui agisce $ G $
		\item 19→ $ G/H $ è sempre {{c1:: omogeneo }} → clz
		\item 20→ Chi è lo stabilizzatore di un $ x\in G/H $? → È in generale un coniugato di $ H $
		\item 21→ Qual è la definizione di funzione equivariante? → È una $ f:X\to Y  $ due $ G $-insiemi t.c. $ f(g\cdot x) =g \cdot f(x) $
		\item 22→ Quando due insiemi sono \textsc{isomorfi} come $ G $-insiemi? → Quando esiste una funzione biunivoca ed equivariante tra loro
		\item 23→ Enuncia il lemma di caratterizzazione degli spazi omogenei → $ G\acts X $ transitiva, preso $ x_0\in X $ e posto $ H :=G_{x_0} $ allora $ X\cong G/H $ come $ G $-insieme 
		\item 24→ La classe dei $ G $-insiemi è una → Categoria, $\Ginsmth$ con\\ $\operatorname{Obj}(\Gins)=\{X\text{ insiemi con una fissata } G \text{-azione}\}$ e \\$ \forall  X,Y $ \Gins :  $ \operatorname{Mor}(X,Y) = \{f:X\to Y \text{ equivarianti}\} $
		\item 25→ Enuncia che relazione collega lo stabilizzatore di un elemento con la sua orbita → $ Gx \cong G/G_x $ come \Gins
		\item 26→ Qual è la definizione di punto fisso di un'azione? → se $G\acts X  $, un \textsc{punto fisso }dell'azione è un $ x\in X $ t.c $ g\cdot x = x \forall g\in G $
		\item 27→ $ x $ è un punto fisso $ \Leftrightarrow $ {{c1::$ G_x = G $ }}  → clz
		\item 28→ Qual è la definizione di azione fedele? → Posto $ \alpha:G \to S_X  $ il morfismo associato a $ G\acts X $, dico che l'azione è \textsc{fedele} quando $ \alpha $ è iniettivo
		\item 29→ Data un'azione $ \alpha :G\to S_X $ generica, costruisci un'azione fedele. → Dato che $ \alpha $ è un morfismo, $ \exists! \beta : G/\operatorname{Ker}(\alpha) \to S_X$ iniettivo, che induce quindi un'azione fedele.
		\item 30→ Enuncia una caratterizzazione di azione fedele → $ G\acts X $ è fedele $ \iff \forall g\in G\setminus{e}\ \exists x\in X : g\cdot x \neq x$
		\item 31→ È $ GL(V)\acts \mathbb{P}(V) $ effettiva? → No perché $ f := x\to\lambda x $ è t.c. $ f\neq id_V $ e $ \alpha(f) = id_V $
		\item 32→ Se $ G\acts X $ qual è la definizione di punto io di un $ g\in G $? → $ G\acts X $, dico che $ x\in X $ è \textsc{punto fisso} di $ g\in G $ quando $ g\cdot x = x $.
		\item 33→ Qual è la definizione di azione libera? → $ G\acts X $ è \textsc{libera} quando $ \forall g\in G, g\neq e$ vale che  $ g$ non ha punti fissi 
		\item 34→ Enuncia una caratterizzazione di azione fedele → $ G\acts X $ è \textsc{libera} $ \iff\ \forall g\in G\setminus{e}, \forall x\in X: g\cdot x \neq x $ 
		\item 35→ Qual è la definizione di sistema di rappresentanti? → $ X $ insieme, $ \sim  $ relazione di equivalenza su $ X $, dico \textsc{sistema di rappresentanti} un insieme $ S\subseteq X $ t.c. $ \pi_{|_{S}}:S\to X/\sim $ (proiezione canonica) è biettiva.
		\item 36→ Enuncia l'equazione delle orbite → $ X,G $ finiti, $ G\acts X $, sia $ S={x_1,...,x_k} $ un sistema di rappresentanti per la relazione \enquote*{essere nella stessa orbita}, allora $ \#X = \sum_{i=1}^k \#G/\#G_{x_i} $
		\item 37→ Enuncia cosa è l'azione per traslazione → È l'azione $ G\acts G:\ g\cdot x:= gx $
		\item 38→ Enuncia qual è l'azione per moltiplicazione a destra. → 	È l'azione $ G\acts G: g\cdot x := xg^{-1} $
		\item 39→ Qual è la definizione di azione destra? → È una funzione $ X\times G\to X: (x,g)\mapsto x\cdot g $ che verifica $ x\cdot  e = x;\ \ (x\cdot g_1)\cdot g_2 = x\cdot (g_1g_2) $
		\item 40→ Come posso passare da un'azione destra ad un'azione sinistra (o viceversa?) → Se $ B:X\times G\to X $ è un'azione destra, allora $ A:G\times X\to X: A(g,x):=B(x,g^{-1}) $ è un'azione sinistra
		\item 41→ Enuncia qual è l'azione di $ G\acts G $ per coniugio → È l'azione $ G\acts G:\ g\cdot x := gxg^{-1} $
		\item 42→ Qual è la definizione di automorfismo interno? → È un morfismo della forma: dato $ \operatorname{inn}_g:G\to G:x\mapsto gxg^{-1} $ per un certo $ g\in G $
		\item 43→ Qual è il morfismo associato all'azione $ G\acts G $ per coniugio? → È $ \alpha:G\to S_G:\alpha(g) = \operatorname{inn_g} $
		\item 44→ È l'azione per coniugio $ G\acts G $ libera? → No: $ \operatorname{inn}_g(e) = e \forall g $, $ e $ è un punto fisso dell'azione
		\item 45→ Qual è la definizione di $ X^G $ se $ G\acts X $? → $ X^G := \operatorname{Fix}_G(X):=\{x\in X\mid \forall g: g\cdot x=x\}$
		\item 46→ Se $ G\acts G $ per coniugio, chi è $ \operatorname{Fix}_G(G) $? → $ \operatorname{Fix}_G(G) = Z(G) $ il centro di $ G $ 
		\item 47→ Se $ G\acts G $ per coniugio, chi è $ G_x $? →  $ G_x = Z_G(x) $ il centralizzante di $ x $
		\item 48→ Enuncia l'\textsc{equazione delle classi} → È $ |G| = |Z(G)|+\sum_{i=1}^m [G:Z_G(x_i)] $ se $ x_1,...,x_k  $ è un sistema di rappresentanti della relazione delle orbite, con $ x_1,...,x_m \not \in Z(G)$
		\item 49→ Qual è la definizione di $ p $-gruppo? → È un gruppo $ G $ con $ o(G)=p^\alpha$, $ p $ primo e $ \alpha \in \N^{>0} $ 
		\item 50→ Com'è il centralizzante di un $ p $ gruppo? → È banale, $ Z(G) =\{1\} $
		\item 51→ Puoi estendere un'azione su un insieme al suo insieme delle parti? → Sì, è facile vedere che se $ G\acts X $, allora si ha anche che $ G\acts \mathcal{P}(X):\ g\cdot E := \{g\cdot x\mid x\in E\}(=\alpha(g)(E)) $ 
		\item 52→ Come puoi far agire un gruppo $ G $ sull'insieme dei suoi sottogruppi? → Sapendo che $ G\acts \mathcal{P}(G) $ per coniugio, ho che $ \mathcal{S}(G):=\{\text{sottogruppi di} G\}$ è un insieme invariante. Allora posso restringere l'azione sopra $ g\cdot H := gHg^{-1} $
		\item 53→ Presa l'azione $ G\acts S(G) $ per coniugio, e $ H\in S(G) $, chi è $ G_H $? → È il \textsc{normalizzante} di $ H$ in $ G $  
		\item 54→ Presa l'azione $ G\acts S(G) $ per coniugio dai una caratterizzazione dell'orbita di $ H<G $ → $ GH=\{\text{coniugati di }H\} $, $ |GH| = |G|/|G_H|=\#\textsc{coniugati di }H= |G|/N_G(H) $
		\item 55→ che relazione c'è tra un sottogruppo $ H<G $ e il suo normalizzante? → $ H\lhd N_G(H) $ e $ H\lhd G \iff N_G(H) = G $
		\item 56→ Come puoi descrivere euristicamente il normalizzante di un sottogruppo $ H< G $ ? → È il \enquote*{più grande sottogruppo di $ G $ in cui $ H $ è normale }, cioè se $ H'<G,\ H\subseteq H', H\lhd H' \then H'\subseteq N_G(H)$
		\item 57→ Che rapporto c'è tra il centro di un elemento $ x\in G $ ed il normalizzante del suo gruppo generato? → Vale in generale $ Z_G(x)\subseteq N_G(<x>) $, non vale in generale l'uguaglianza
		\item 58→ Enuncia il teorema di Cayley nel contesto delle azioni → Ogni gruppo ha un'azione fedele su un qualche insieme
		\item 59→ Esibisci un Gruppo che non possiede sottogruppi di un determinato ordine che divide l'ordine del gruppo → $ G=A_4 $ nonn possiede sottogruppi di ordine $ 6 $
		\item 60→ Qual è l'ordine di un generico elemento di un gruppo ciclico? → Se $ G= <g>,\ o(G) =  n$, allora $ o(g^s) = \frac{n}{(n,s)} $ 
		\item 61→ Quanti generatori ha un gruppo ciclico $ G=<g>\ o(G) = n $? → Sono tanti quanto i naturali $ \leq n $ che non dividono $ n $, cioè $ \varphi(n) $
		\item 62→ Caratterizza i sottogruppi di ordine $ d|n $ di un gruppo ciclico $ G $ → Sono gli $ H<G $ t.c. $ H=<g^{n/d}> $, questo esiste ed unico $ \iff d|n$ 
		\item 63→ Enuncia la \textsc{formula di Gauß} → $ \sum\limits_{d:d|n} \varphi(d) = n $
		\item 64→ Dai una condizione sui sottogruppi di un gruppo $ G $ sufficiente affinché esso sia ciclico → $ G $ gruppo, $ o(G) = n  $, se $ \forall d|n $ esiste al più un sottogruppo di ordine $ d $, allora $ G $ è ciclico 
		\item 65→ Dai la definizione di $G^d$ e una condizione su di ess affinché il gruppo $G$ sia ciclico → $G^d = \{ x\in G\mid x^d = 1\}$. Se $\forall d \mid n:\ |G^d|\leq d \then\ G$ è ciclico. 
		\item 66→ Dai delle condizioni sufficienti per trovare dei sottogruppi di un gruppo abeliano → $ G $ gruppo \emph{abeliano}, $o(G) = n,\ d|n\then\ \exists H\leq G:\ o(H)=d$
		\item 67→ Cosa puoi dire dei sottogruppi di gruppi ciclici e abeliani? → \begin{enumerate}
			\item $ G $ abeliano $ \then\ \forall d|n\ \exists  $ un sottogruppo di $ G  $ di ordine $ n $ 
			\item $ G $ ciclico $ \then\ \forall d|n\ \exists  $ un sottogruppo di $ G  $ di ordine $ n $ 
		\end{enumerate}
		\item 68→ Enuncia il teorema di Sylow→ Dato un gruppo finito $ G $, posto $ o(G) = p^\alpha m$, $ p $ primo, $ p\nmid  m$, allora \begin{enumerate}
			\item $ \exists $ un $ p $-sylow in $ G $
			\item se $ H\leq G $ è un $ p $-sottogruppo $ \then\ H $ è contenuto in un $ p $-sylow
			\item tutti i $ p $-sylow sono coniugati
			\item Detto $ n_p=\#\{p-\text{sylow}\}\ \then\ n_p = [G:N_G(P)]$, con $ P $ un $ p $-sylow.
			\item $ n_p \equiv 1 \ \operatorname{mod} p\ $ e $ n_p\mid n $
		\end{enumerate}
		\item 69→ Enuncia il lemma che ti permette di trovare dei $ p $-sylow di un sottogruppo → $ G $ gruppo finito, $ H $ sottogruppo e $ P p$-sylow $ \then\ \exists x\in G:\ xPx^{-1}\cap H  $ è un $ p $-sylow di $ H $ 
		\item 70→ Dati $ P,H<G $, come può agire $ P\times H  $ su $ G $ ? Di' chi è lo stabilizzatore di un elemento→ $ P\times H \acts G:\ (a,h)\cdot x = axh^-1 $, e inoltre $ (P\times H)_x \cong (x^{-1}Px)\cap H$ tramite $ f $: $ f(a,h) = h $ e $ f^{-1}(h)=(x^{-1}Px,h) $
		\item 71→ Enunciai il corollario del lemma sui $ p $-sylow dei sottogruppi → $ H\leq G $, $ G $ ha un $ p $-sylow $ \then\  $ anche $ H $ ha un $ p $-sylow
		\item 72→ Dai un'idea di quale sia la strategia per dimostrare il teorema di Sylow → Eseguo due passi: \begin{enumerate}
			\item trovo una classe di gruppi con un \psy banale, cioè $ GL(n,{F}_p) $, $ p $ primo e $ F_p=\Z/p $
			\item esibisco un morfismo iniettivo in $GL(n,{F}_p) $ per un qualsiasi gruppo finito
		\end{enumerate}
		\item 73→ Qual è la cardinalità di $ GL(n,F_p) $ con $ p  $ primo e $ F_p=\Z/p $? → È $ \prod_{i = 0}^{n-1}(p^n-p^i) $
		\item 74→ Enuncia una caratterizzazione dei gruppi di ordine un quadrato di un primo → $ G $ gruppo, $ Z(G)\neq \{1\},\ \# G = p^2,\ p$ primo $ \then G\cong C_{p^2} $ oppure $ G\cong C_p\times C_p $
		\item 75→ Enuncia una condizione sufficiente su un gruppo affinché esso sia abeliano che va a guardare $ G/Z(G) $ → $ G $ finit. $ G/Z(G) $ ciclico $ \then\ G $ abeliano
		\item 76→ Esibisci il morfismo iniettivo da $ S_n\hookrightarrow GL(n,F_p) $ → È il morfismo che manda $ \sigma \mapsto \varphi(\sigma):= (e_{\sigma(1)}|...|e_{\sigma(n)}) $ cioè che manda un morfismo nella matrice identità con le colonne permutate.
		\item 77→ Descrivi il funtore dalla categoria degli insiemi alla categoria degli spazi vettoriali su un campo fissato → È una mappa della forma $ \mathbf{Set}\overset{\alpha}{\to} \mathbf{Vec}(F): \operatorname{Mor}_{\mathbf{Set}}(X,Y) = \{f:X\to Y\}\overset{\alpha}{\to}\{f^*:F^Y\to F^X\} = \operatorname{Mor}_{\mathbf{Set}}(\alpha(X),\alpha(Y)):f\mapsto f^* $, con $ f^*:F^Y\to F^X: u\mapsto u\circ f $ È un funtore controvariante
		\item 78→ Descrivi il processo di linearizzazione di un'azione → ???
		\item 79→ Qual è la definizione di $ F $-\textsc{algebra} ? → È un anello $ A $ che è anche un $ F $-spazio vettoriale, con la stessa struttura additiva e t.c. $ \forall a,b \in A\forall \lambda \in F:\ \lambda(ab) = a(\lambda b) = (\lambda a )b $
		\item 80→ Esibisci una $ F $-algebra facile → Dati $ X $ insieme e $ F $ campo, $ F^X $ è un'$ F $-algebra commutativa con unità $ 1:X\to F:x\mapsto 1\in F $
		\item 81→ Dato un $ p$-gruppo, esibisci dei suoi sottogruppi → $ \# G = p^\alpha,\ p $ primo $ \then\ \forall i = 0,...,\alpha\ \exists H<G:\ \# H = p^i $
		\item 82→ Enuncia il lemma che caratterizza il prodotto$ NH $ di due sottogruppi $ N,H<G $, data la funzione $ f:N\times H\to G,\ f(n,h):= nh $ → \begin{enumerate}[noitemsep]
			\item $\operatorname{Im}(f) = NH  $
			\item $ nh\in \operatorname{Im}(f) $ e $ f^{-1}(nh)=\{(nx,x^{-1}h)\mid x\in N\cap H \} $
			\item $ |NH| = |N||H|/|N\cap H| $
			\item $ N\lhd G\then\ NH<G $
			\item $ N,H\lhd G\then\ NH\lhd G $
			\item Se $ N,H\lhd G\then\ [N,H]\subseteq N\cap H $
			\item Se $ H,N\lhd G, N\cap H = \{1\}\ \then\ NG\cong N\times H $
		\end{enumerate}
		\item 83→ Enuncia la proposizione sull'equivalenza tra il prodotto diretto interno ed esterno → $ H,G\lhd G, N\cap H = \{1\}, NH = G\then\ N\cong N\times H $.\\ D'altro lato, se pongo $ N\times H=: K  $ ho: $ \bar N := N\times \{1\} \lhd K,\ \bar H :=\{1\}\times H\lhd K $ sono tali che $ \bar N \cap \bar H = \{(1,1)\} e K=\bar N \bar H $
		\item 84→ Enuncia il lemma di caratterizzazione del prodotto di un numero finito arbitrario di gruppi → $ G $ gruppo, $ N_1,...,N_k\lhd G $ tali che $ N_i \cap (N_1\cdot ... \cdot \hat N_i \cdot ... \cdot N_k)  = \{1\}\ \then\ f:N_i\times ...\times N_k \to N_1 \cdot ... \cdot N_k : (n_1, ..., n_k)\to n_1\cdot ...\cdot n_k  $ è un isomorfismo
		\item 85→ Enuncia il teorema di Cauchy per gruppi → $ G $ gruppo finito, $  p $ primo t.c. $ p| o(G)\then\ G $ contiene un elemento di ordine $ p $ 
		\item 86→ Enuncia il teorema di caratterizzazione dei gruppi con \psy unici → $ G  $ gruppo finito, $ o(G)= p_1^{\alpha_1}...p_k^{\alpha_k},$  $\ p_i $ primi, se tutti i \psy sono unici $ \then\ $ posti $ P_1,...,P_k $ gli unici \psy ho che $ G \cong P_1\times...\times P_k $
		\item 87→ Cosa puoi dire su $ n_p  $ se il \psy è normale? → Ho che $ P\lhd G\ \iff\ n_p = 1$
		\item 88→ Enuncia la proposizione di caratterizzazione dei gruppi con ordine un prodotto di primi → $ G $ gruppo con $ o(G) = pq $, $p,q$ primi, $ p<q$ e $ p\nmid q-1\ \then\  G\cong C_p\times C_q (\cong C_{pq})$
		\item 89→ 
		\item 90→ 
		\item 91→ 
		\item 92→ 
		\item 93→ 
		\item 94→ 
		\item 95→ 
		\item 96→ 
		\item 97→ 
		\item 98→ 
		\item 99→ 
		\item 100→ Cos'è la caratteristica di un anello? → Preso un anello $ A $ esiste sempre un morfismo $ \phi:\Z \to A: n\mapsto \phi(n) = n\cdot 1_A $. Essendo $ \Z $ un PID, il nucleo di $ \phi $ sarà principale, cioè esiste un $ n $ t.c. $ \ker\phi= (n) $. Questo $ n $ è la \textsc{caratteristica} dell'anello $ A $.
		\item 101→ Come si comporta il prodotto in un campo di caratteristica $ n $ ? → preso $ m\in \Z $ e $ x\in K^* $, $ m\cdot x  = 0$ sse $ n|m $ con $ n = \operatorname{car}K$ 
		\item 102→ Qual è la definizione di campo primo? → È il campo generato dall'immagine di $ \phi:n\mapsto n\cdot 1_A $
		\item 103→ Qual è la definizione di estensione di campi? → È un morfismo di anelli $ i:F\to E $ dove $ E,F $ sono campi. Si verifica che è automaticamente iniettiva.
		\item 104→ Se $ E/F $ è un'estensione di campi, $ E $ è {{c1:: uno spazio vettoriale su }} $ F $ → clz
		\item 105→ Dato un campo $ F $, cos'è una $ F $-algebra? → È un anello $ A $ con una \enquote*{moltiplicazione per scalare} $ F\times A\to A $ che rende $ A $ uno spazio vettoriale su $ F $; inoltre deve valere che \[\lambda(ab)=a(\lambda b) = (\lambda a)b\]
		\item 106→ Qual è la definizione di grado di un'estensione $ E/F $? È il numero $ [E:F] = \dim_F E$, dimensione come spazio vettoriale.
		\item 107→ Qual è la definizione di estensione finita? → È un'estensione $ E/F $ per la quale $ [E:F]<\infty$
		\item 108→ Quando la mappa che manda un polinomio nella funzione indotta dal polinomio è iniettiva? → Quando il campo è infinito
		\item 109→ Data un'estensione di campi $ E/F $, qual è la definizione di campo intermedio? → È un campo $ K $ tale che $ F\subset K\subset E $
		\item 110→ Qual è la definizione di campo generato da $ S\subset E $? → È l'intersezione di tutti i campi intermedi che contengono $ S $
		\item 111→ Qual è la definizione di sistema di generatori di un'estensione $ E/F $? È un insieme $ S\subset E $ tale che $ E = F(S)$
		\item 112→ Qual è la definizione di estensione finitamente generata? → È un'estensione $ E/F $ per cui esiste un insieme finito $ S $ tale che $ E=F(S) $
		\item 113→ Qual è la definizione di estensione semplice? → È un'estensione generata da un solo elemento
		\item 114→ Se $ E/F $ e $ F/K $ sono estensioni finite, allora $ [E:K] =  $ ? → $ = [E:F]\cdot [F:K] $
		\item 115→ Come costruisco un'estensione che contenga una radice di un polinomio irriducibile? → $ F $ campo, $ f $ irriducibile, allora la composizione $ F\hookrightarrow F[X] \to K:=F[X]/(f)$ è un'estensione di grado $ [K:F] = \deg f $. Se $ \pi $ è la proiezione canonica, allora $ \gamma:= \pi(X)\in K $ è una radice di $ f $, e $ K=F(\gamma) $
		\item 116→ Enuncia il procedimento di Kronecker → Sia $ F $ un campo e $ f $ un polinomio di grado $ d\geq 1 $. Allora c'è un'estensione finita di $ F $ in cui $ f $ possiede una radice. (Si prende la proposizione che richiede che $ f $ sia un polinomio irriducibile e la si applica ad un fattore irriducibile di $ f $) 
		\item 117→ Qual è la definizione di numero algebrico di un'estensione? → Data $ E/F $ dico che $ \alpha \in E $ è \textsc{algebrico } su $ F $ se esiste un polinomio $ p(X)\in F[X] $ tale che $ p(x)\not \equiv 0, p(\alpha) = 0 $. Un elemento non algebrico è \textsc{trascendente}
		\item 118→ Data $ E/F $ estensione, qual è la definizione di polinomio minimo di $ \alpha \in E $? → È il \enquote*{generatore monico dell'ideale $ \ker v_\alpha $}, cioè un polinomio monico che divide ogni polinomio che ha soluzione $ \alpha $. Si indica con $ m_{\alpha,F}$,o $m_\alpha  $
		\item 119→ Enuncia la proposizione sulla relazione tra $ F[x]/(f) $ e $ F(\alpha) $ dove $ E/F $ estensione $ f = m_\alpha $ e $ \alpha $ algebrico → Data $ E/F $ un'estensione, $ \alpha $ algebrico e $ f=m_\alpha $ polinomio minimo. Allora la valutazione induce un isomorfismo \[ \varphi_\alpha: F[X]/(f) \overset{\cong}{\longrightarrow} F(\alpha)\ \ \  \varphi_\alpha (g+(f)) =g(\alpha). \] Inoltre $ [F(\alpha):F] = \deg f =:d$ e $ \{1,\alpha,...,\alpha^{d-1}\} $ è una base di $ F(\alpha) $ su $ F $ 
		\item 120→ Per descrivere $ F(\alpha) $ bisogna per forza utilizzare le funzioni razionali? → No è sufficiente valutare i polinomi in quanto $ F(\alpha) = \operatorname{im} \varphi_\alpha = v_\alpha (F[X])$
		\item 121→ Data un'estensione $E/F$, e $\alpha \in E$, allora $\alpha $ è algebrico se e solo se → $[F(\alpha):F]<\infty$ 
		\item 122→ Qual è la definizione di estensione algebrica → è un'estensione in cui ogni elemento è algebrico
		\item 123→ Come vengono caratterizzate le estensioni finit? → Le estensioni finite sono quelle algebriche e finitamente generate o equivalentemente quelle generate da un numero finito di elementi algebrici
		\item 124→ Qual è la definizione di campo algebricamente chiuso? → è un campo per cui ogni polinomio non costante su di esso ammette una radice
		\item 125→ Qual è la definizione di chiusura algebrica di un campo $K$? → è un'estensione algebrica $L/K$ con $K$ algebricamente chiuso
		\item 126→ Enuncia il teorema di Steinitz→ Ogni campo ammette una chiusura algebrica
		\item 127→ Enuncia il lemma sugli elementi algebrici di una chiusura algebrica → $K$ campo, $L/K$ estensione con $L$ algebricamente chiuso, allora $\bar L$
		\item 128→ →
		\item 129→ →
		\item 130→ →
		\item 131→ →
		\item 132→ →
		\item 133→ →
		\item 134→ →
		\item 135→ →
		\item 136→ →
		\item 137→ →
		\item 138→ →
		\item 139→ →
		\item 140→ →
		\item 141→ →
		\item 142→ →
		\item 143→ →
		\item 144→ →
		\item 145→ →
		\item 146→ →
		\item 147→ →
		\item 148→ →
		\item 149→ →
		\item 150→ →
		\item 151→ →
		\item 152→ →
		\item 153→ →
		\item 154→ →
		\item 155→ →
		\item 156→ →
		\item 157→ →
		\item 158→ →
		\item 159→ →
		\item 160→ →
		\item 161→ →
		\item 162→ →
		\item 163→ →
		\item 164→ →
		\item 165→ →
		\item 166→ →
		\item 167→ →
		\item 168→ →
		\item 169→ →
		\item 170→ →
		\item 171→ →
		\item 172→ →
		\item 173→ →
		\item 174→ →
		\item 175→ →
		\item 176→ →
		\item 177→ →
		\item 178→ →
		\item 179→ →
		\item 180→ →
		\item 181→ →
		\item 182→ →
		\item 183→ →
		\item 184→ →
		\item 185→ →
		\item 186→ →
		\item 187→ →
		\item 188→ →
		\item 189→ →
		\item 190→ →
		\item 191→ → 
		\item 192 → 
		\item 193 → 
		\item 194→ 
		\item 195→ 
		\item 196→ 
		\item 197→
		\item 198→ 
		\item 199→ 
		\item 200→
		
	\end{itemize}
	
		
	
\end{document}
